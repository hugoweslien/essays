%! TeX program = xelatex

\documentclass[12pt]{article}

\usepackage[a4paper]{geometry}
\usepackage{verbatim}
\usepackage{ragged2e}
\usepackage{lmodern}
\usepackage{fontspec}
\usepackage{fancyhdr}
\usepackage{setspace}
\setmainfont{Times New Roman}

\pagestyle{fancy}
\fancyhf{}
\renewcommand{\headrulewidth}{0pt}
\fancyfoot[R]{\thepage}

\begin{document} 
\setstretch{1.5}

\begin{titlepage}
    \begin{center}
        \vspace*{\stretch{0.5}}
        \Huge\textbf{Manual de S\'intese de Som}
         
        \vspace{10cm}
    \end{center}
\end{titlepage}

\begin{center}
    \Huge{Introdu\c c\~ao}
\end{center}

\justifying

\pagebreak

\begin{center}
    \Huge{Componentes de Sintetizadores}
\end{center}

\begin{flushleft}
    \Large{I. Oscila\c c\~ao}
\end{flushleft}

\justifying
    Oscila\c c\~ao \'e a produ\c c\~ao de um tipo de onda sonora, que gera sons
    dependendo da forma. As principais ondas sonoras de osciladores s\~ao:
    Sine, Square, Triangle, Sawtooth e Noise. 
    % Inserir imagem aqui 
    Existem 2 formas de gerar sons novas ondas sonoras, adicionando e
    subtraindo. Ao adicionarmos um conjunto de ondas sonoras sine em intervalos
    de frequ\^encia diferentes criamos sons apartir do processo chamado
    s\'intese aditiva; Ao adicionarmos um conjunto de ondas sonoras diferentes e retirando certas frequ\^encias com o filtro criamos sons apartir do processso chamado s\'intese subtractiva. 


\begin{flushleft}
    \Large{II. Filtros}
\end{flushleft}
\justifying
Existem 5 tipos de filtros: Low Pass, High Pass, Band Pass, Notch, Comb. Os
filtros s\~ao usados para cortar certas frequ\^encias apartir do cutoff, que
\'e o ponto em que inicia-se a filtragem. 


\begin{flushleft}
    \Large{III. Envelopes}
\end{flushleft}
\justifying
Este \'e o \'ultimo est\'agio antes do som sair dos sintetizadores paras as
colunas de som, o envelope \'e usado para definir a forma final do som. 
O envelope tem quatro elementos: attack, decay, sustain e
release, comumente chamado de Envelope ADSR. 

\begin{flushleft}
    \large\bf{Attack}
\end{flushleft}

\begin{flushleft}
    \large\bf{Decay}
\end{flushleft}

\begin{flushleft}
    \large\bf{Sustain}
\end{flushleft}

\begin{flushleft}
    \large\bf{Release}
\end{flushleft}

\pagebreak

\begin{flushleft}
    \Large{IV. Modula\c c\~ao (LFO)}
\end{flushleft}

\pagebreak

\begin{center}
    \Huge{Recriando sons}
\end{center}

\justifying
Para recriar sons teremos que criar patches que s\~ao o equivalente a receitas
para sons, usaremos os sintetizadores Analog (plug-in integrado de Ableton
Live) e Xfer Serum. Recriaremos patches para drums, leads, pads, bells, plucks,
arps e fx tendo como refer\^encia os albums: Flume - Skin, Da-P - Shappire, Kanye West -
Yeezus, The Kount - Contact, The Weeknd - Trilogy, Bon Iver - 22 a Million,
Lorde - Melodrama, Kaytranada - 99.9, GEOTHEORY - todos albums , Medasin -
Irene, Stint - todos singles.   
\end{document}





