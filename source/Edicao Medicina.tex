%! TeX program = xelatex
\documentclass[12pt]{article}
\usepackage[a4paper]{geometry}
\usepackage{ragged2e}
\usepackage{lmodern}
\usepackage{fontspec}
\usepackage{fancyhdr}
\usepackage{setspace}
\setmainfont{Times New Roman}

\pagestyle{fancy}
\fancyhf{}
\renewcommand{\headrulewidth}{0pt}
\fancyfoot[R]{\thepage}

\begin{document} 
\setstretch{1.5}

\begin{titlepage}
    \begin{center}
        \vspace*{\stretch{0.5}}
        \Huge\textbf{Modelo de Pesquisa}
         
        \Large\textit{Edi\c c\~ao Medicina}
         
        \vspace{10cm}
    \end{center}
\end{titlepage}

% Introduction 
\begin{center}
    \Huge\textbf{Introdu\c c\~ao}
\end{center}

% Content
\justifying
Todos dias pessoas encontram-se num cen\'ario, em que t\^em de decidir que
dieta devem seguir, que exerc\'icios f\'isicos devem fazer, como os
exerc\'icios devem ser
feitos,  que suplementos
devem tomar, etc. Para esclarecer estas quest\~oes, pessoas recorrem a blogs,
livros, v\'ideos na internet, familiares, colegas e por fim  m\'edicos. Neste
processo, elas encontram v\'arias
opini\~oes e por vezes umas em conflito com as outras, dificultando a tomada
de decis\~ao. 
No final elas escolhem as recomenda\c c\~oes mais
convincentes, o problema \'e que nem sempre esta \'e a melhor forma de lidar
com a incerteza, porque se decidirmos fazer algo com base em opini\~oes
sem fundamento e sem a devida pesquisa corremos o risco de nos prejudicar a longo prazo. 

O ideal seria termos um modelo de tomada de decis\~ao que ajuda-nos a avaliar as
cren\c cas que temos e as cren\c cas que adoptamos dos outros, mas isso fica
para o pr\'oximo ensaio, em vez disso, para responder as quest\~oes de sa\'ude que
tivermos, iremos desenvolver um modelo de pesquisa.  
Pense neste modelo como  uma camada de protec\c c\~ao ou o que programadores chamariam de \textit{debugger}, uma camada para remover erros; neste caso erros de julgamento.  
Para minimizar os erros precisamos de uma fonte de verdade, duas fontes v\^em
em mente: experi\^encia e artigos cient\'ificos. Daremos prioridade a
experi\^encia por ser a fonte que nos dir\'a se o nosso corpo est\'a a reagir
bem diante das mudan\c cas que introduzirmos ou remorvemos na nossa
vida. Em segundo lugar temos a colec\c c\~ao de artigos cient\'ificos
publicados nos jornais de medicina, apresentados nos livros e blogs de sa\'ude que servir\~ao de
guia para analisarmos e testarmos as nossas e as opini\~oes dos outros. 
%\noindent
\pagebreak


\begin{center}
    \Huge\textbf{Fonte Veritatis}
\end{center}

\end{document}




