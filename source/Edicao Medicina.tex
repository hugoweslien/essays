%! TeX program = xelatex
\documentclass[12pt]{article}
\usepackage{geometry}
\usepackage{lmodern}
\usepackage{fontspec}
\setmainfont{Times New Roman}


\begin{document} 

\begin{titlepage}
    \begin{center}
        \vspace*{\stretch{0.5}}
        \Huge\textbf{M\'odulos de Pesquisa}
         
        \vspace{10cm}
    \end{center}
\end{titlepage}

% Introduction 
\begin{center}
    \Huge\textbf{Introdu\c c\~ao}
\end{center}

% Content
Quando ficamos doentes uns dos primeiros conselhos que recebemos \'e que
devemos ir ao hospital e uma vez no hospital depois de marcarmos a consulta,
fazermos testes, os m\'edicos nos apresentam os resultados dos testes e o
tratamento que devemos seguir para resolver o problema.
Embora confiemos no julgamento dos m\'edicos, uma quest\~ao que n\~ao colocamos
\'e: Que tal as observa\c c\~oes do m\'edico est\~ao erradas?

M\'edicos s\~ao seres humanos , \'e como qualquer um de
n\'os suscept\'iveis \`a erros. Se o diagn\'ostico que nos foi apresentado for
errado corremos o risco de tomar decis\~oes co base numa cren\c ca/ hip\'otese
falsa.
A consequ\^encia num caso extremo ser\'a morte e num menos
extremmo ser\~ao efeitos dos medicamentos como: dores de cabe\c a, v\^omitos,
fadiga,etc.

Para resolvermos est\'a quest\~ao precisamos de mais uma camada de confian\c
ca, neste caso a habilidade de avaliar a literatura cient\'ifica, isto \'e o
que a comunidade cient\'ifica concorda em rela\c \~ao a que tratamentos
funcionam e que tratamentos n\~ao funcionam.

%\pagebreak

\end{document}


