%! TeX program = xelatex
\documentclass[12pt]{article}
\usepackage{geometry}
\usepackage{lmodern}
\usepackage{fontspec}
\usepackage{fancyhdr}
\setmainfont{Times New Roman}


\pagestyle{fancy}
\fancyhf{}
\renewcommand{\headrulewidth}{0pt}
\fancyfoot[R]{\thepage}

\begin{document} 

\begin{titlepage}
    \begin{center}
        \vspace*{\stretch{0.5}}
        \Huge\textbf{Modelo de Pesquisa}
         
        \Large\textit{Edi\c c\~ao Medicina}
         
        \vspace{10cm}
    \end{center}
\end{titlepage}

% Introduction 
\begin{center}
    \Huge\textbf{Introdu\c c\~ao}
\end{center}

% Content
Todos dias pessoas encontram-se num cen\'ario em que t\^em de tomar decis\~oes
relaccionadas \`a sa\'ude, que dieta devem seguir, que exerc\'icios f\'isicos
devem fazer, que suplementos devem tomar, etc. Para esclarecer essas d\'uvidas
recorrem a blogs, livros, v\'ieos na internet, familiares, colegas e na
\'ultima das inst\^ancias m\'edicos. Neste processo, encontram v\'roas
opini\~oes e muitas vezes umas em conflito com as outras, o que torna a tomada
de decis\~ao ainda mais dif\'icil.
No final as sugest\~oes/ opini\~oes mais convincentes acabam sendo escolhidas
mas essa nem sempre \'e a melhor forma de lidar com este tipo de situa\c c\~ao
pois se formos a tomar decis\~oes baseadas em sugest\~oes sem fundamento
corremos o risco de nos prejudicar a longo prazo.  

O ideal seria criar um modelo de tomada de decis\~ao que ajuda-nos a avaliar
nossas decis\~oes com base nas nossas cren\c cas e as cren\c cas dos outros. 
Mas n\~ao iremos por esse caminho hoje, em vez disso vamos adoptar um modelo de
pesquisa relaccionado a quest\~oes de sa\'ude, isto \'e, depois de irmos ao
m\'edico e esclarecermos as nossas d\'uvidas iremos usar este modelo para
testar as hip\'oteses que tinhamos antes de ir ao m\'edico e as hip\'oteses
apresentadas pelo m\'edico.  
Pense nisto como mais uma camada de confian\c ca ou
o que programadores chamariam de \textit{debugger}, uma camada para remover
erros; neste caso erros de julgamento.  Para esta tarefa a nossa fonte de verdade ser\'a a cole\c c\~ao de artigos
cient\'ificos publicados nos jornais cient\'ificos de medicina ou apresentados
nos livros e blogs de sa\'ude.  
%\noindent
\pagebreak


\begin{center}
    \Huge\textbf{Fonte Veritatis}
\end{center}

\end{document}




