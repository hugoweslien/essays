%! TeX program = xelatex

\documentclass[12pt]{article}

\usepackage[a4paper]{geometry}
\usepackage{verbatim}
\usepackage{graphicx}
\graphicspath{ {./images/} }
\usepackage{ragged2e}
\usepackage{lmodern}
\usepackage{fontspec}
\usepackage{fancyhdr}
\usepackage{setspace}
\setmainfont{Times New Roman}

\pagestyle{fancy}
\fancyhf{}
\renewcommand{\headrulewidth}{0pt}
\fancyfoot[R]{\thepage}

\begin{document} 
\setstretch{1.5}

\begin{titlepage}
    \begin{center}
        \vspace*{\stretch{0.5}}
        \Huge\textbf{Modelo de Pesquisa}
         
        \Large\textit{Edi\c c\~ao Medicina}
         
        \vspace{10cm}
    \end{center}
\end{titlepage}

\begin{center}
    \Huge{Introdu\c c\~ao}
\end{center}

\justifying
Todos temos um tempo limitado neste planeta, a dura\c c\~ao varia com os
cuidados que temos com o nosso corpo. No entanto, se olharmos para a forma como
n\'os, nossos familiares, colegas e amigos tratamos os nossos corpos
concluiremos que existe muito espa\c co para melhorias. Este espa\c co
continuar\'a a existir se mantivermos as mesmas escolhas, h\'abitos e cren\c
cas,  levando-nos \`a um futuro em que a nossa expectativa de vida estar\'a
reduzida e ao longo do caminho teremos os efeitos secund\'arios dessas escolhas
(como baixa energia, press\~ao alta, problemas de coluna, barriga de cerveja
entre outras).

Para evitar que isto aconten\c ca,  temos que assumir o controle, como diz o
Tomassi (2011) "o poder real \'e o grau em que uma pessoa tem controle sobre as
suas pr\'oprias circunst\~ancias. O poder real \'e o grau em que controlamos as
direc\c c\~oes das nossas vidas". Isto significa que assumiremos o controle das
pesquisas, experi\~encias, monitoramento dos biomarcadores, mensuramento do
progresso e acima de tudo remover a ideia de que s\'o m\'edicos \'e que
sabem o que \'e saudavel para n\'os; teremos aux\'ilio de m\'edicos mas as
escolhas ser\~ao nossas no fim do dia. 

Para esta tarefa proponho um modelo de pesquisa que envolve 3 est\'agios: a
recolha de hip\'oteses (atrav\'es de artigos cient\'ificos), an\'alise dos
tratamentos (usando m\'etodos estat\'isticos e de racionalidade) e
experi\^encias (onde biomarcadores e como nos sentimos ajudaram-nos a
determinar se o tratamento \'e eficaz). 
\pagebreak


\begin{center}
    \Huge{Anatomia e Fisiologia Humana}
\end{center}

\justifying
A biologia \'e a ci\^encia que estuda a vida e os organismos vivos, fazem parte
dos ramos da biologia a anatomia e fisiologia. A anatomia estuda como o corpo
est\'a estruturado (as partes) e a fisiologia estuda como as partes do corpo 
funcionam.  O corpo humano \'e uma estrutura complexa de c\'elulas, tecidos, 
org\~aos e sistemas que trabalham em conjunto para manter-nos vivos. 
Esta estrutura complexa \'e composta por 10 sistemas biol\'ogicos: esquel\'tico,
cardiovascular, endocrino, respirat\'orio, digestivo, urin\'rio, reprodutivo e
limf\'tico, cada um executando uma fun\c c\~ao para manter o nosso corpo em
equil\'ibrio (homeostase).

\begin{flushleft}
\Large{1. Sistema Esquel\'etico}
\end{flushleft}
\justifying
\noindent
Pare por um momento e tente imaginar-se sem ossos, serias uma massa de
org\~aos, pele e m\'usculos. Para garantir que esses elementos tenham forma
precisam de ossos, que fazem parte do sistema esquel\'etico assim como
cartilagem,  ligamentos e outros tecidos. Dando-nos suporte, facilitando
locomo\c c\~ao e protegendo os org\~aos internos. 

\begin{flushleft}
\Large{2. Sistema Nervoso}
\end{flushleft}
\justifying
\noindent
O sistema nervoso colecta informa\c c\~ao do nosso ambiente atrav\'es dos 5
sentidos: a vis\~ao, o olfato, o paladar, a audi\c c\~ao e o tato, processa a
informa\c c\~ao atrav\'es do c\'erebro e envia um sinal para as outras partes
do corpo.  Um exemplo: Ao tocarmos uma ch\'avena de ch\'a quente, estamos a
usar o tato, o cer\'ebro processa a temperatura da ch\'avena e conclui que a
temperatura \'e muito alta para as nossas m\~aos e manda um sinal para os dedos
largarem a ch\'avena. 


\begin{figure}[h]
\includegraphics[scale=0.5]{bodysys.jpeg}
\caption{Sistemas do Corpo Humano}
\end{figure}
\pagebreak



\begin{center}
    \Huge{Biomarcadores}
\end{center}
\justifying
Antes de alterar certos factores do nosso estilo de vida temos que 
ter um ponto
de refer\^encia que serve como base para medirmos o progresso. Iremos
estabelecer esta base olhando para os exames de sangue que cont\'em os
indicadores mensur\'aveis do nosso estado biol\'ogico (biomarcadores).
Avaliaremos os seguintes biomarcadores: Testosterona, Hemot\'ocrito,
Fibrinog\'enio, Prote\'ina C-Reactiva, Magn\'esio, Ferro, Ferritina, Vitamina
B12, Colesterol, HDL, LDL, HDL-P, LDL-P, Lp(a), Triglic\'erides, Glucose em
jejum, Toler\^ancia da Glucose Oral, HbA1c, Insulina, Estradiol, Glutatione,
Vitamina D. 



\begin{flushleft}
\Large{1. Testosterona}
\end{flushleft}






\pagebreak

\begin{center}
    % A tabela com a descricao dos biomarcadores a os intervalos
    \Huge{Estabelecendo a Base}
\end{center}


\end{document}




