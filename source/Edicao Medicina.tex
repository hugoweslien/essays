%! TeX program = xelatex

\documentclass[12pt]{article}

\usepackage[a4paper]{geometry}
\usepackage{ragged2e}
\usepackage{lmodern}
\usepackage{fontspec}
\usepackage{fancyhdr}
\usepackage{setspace}
\setmainfont{Times New Roman}

\pagestyle{fancy}
\fancyhf{}
\renewcommand{\headrulewidth}{0pt}
\fancyfoot[R]{\thepage}

\begin{document} 
\setstretch{1.5}

\begin{titlepage}
    \begin{center}
        \vspace*{\stretch{0.5}}
        \Huge\textbf{Modelo de Pesquisa}
         
        \Large\textit{Edi\c c\~ao Medicina}
         
        \vspace{10cm}
    \end{center}
\end{titlepage}

% Introduction 
\begin{center}
    \Huge\textbf{Introdu\c c\~ao}
\end{center}

% Content
\justifying
% Shared Context
Todos temos um tempo limitado neste planeta, a dura\c c\~ao varia com os
cuidados que temos com o nosso corpo. No entanto, se olharmos para a forma como
n\'os, nossos familiares, colegas e amigos tratamos os nossos corpos
concluiremos que existe muito espa\c co para melhorias. Este espa\c co
continuar\'a a existir se mantivermos as mesmas escolhas, h\'abitos e cren\c
cas,  levando-nos \`a um futuro em que a nossa expectativa de vida estar\'a
reduzida e ao longo do caminho teremos os efeitos secund\'arios dessas escolhas
(como baixa energia, press\~ao alta, problemas de coluna, barriga de cerveja
entre outras).

Para evitar que isto aconten\c ca,  temos que assumir o controle, como diz o
Tomassi (2011) "o poder real \'e o grau em que uma pessoa tem controle sobre as
suas pr\'oprias circunst\~ancias. O poder real \'e o grau em que controlamos as
direc\c c\~oes das nossas vidas". Isto significa que assumiremos o controle das
pesquisas, experi\~encias, monitoramento dos biomarcadores, mensuramento do
prpogresso e acima de tudo remove a ideia de que s\'o so m\'edicos \'e que
sabem o que \'e saudavel para n\'os; teremos aux\'ilio de m\'edicos mas as
escolhas ser\~ao nossas no fim do dia. 

Proponho para esta tarefa um modelo de pesquisa que envolve 3 est\'agios: a
recolha de hip\'oteses (atrav\'es de artigos cient\'ificos), an\'alise dos
tratamentos (usando m\'etodos estat\'isticos e de racionalidade) e
experi\^encias (onde biomarcadores e como nos sentimos ajudaram-nos a
determinar se o tratamento \'e eficaz). 
\pagebreak


\begin{center}
    \Huge\textbf{Fonte Veritatis}
\end{center}

\end{document}




