%! TeX program = xelatex

\documentclass[12pt]{article}

\usepackage[a4paper]{geometry}
\usepackage{ragged2e}
\usepackage{lmodern}
\usepackage{fontspec}
\usepackage{fancyhdr}
\usepackage{setspace}
\setmainfont{Times New Roman}

\pagestyle{fancy}
\fancyhf{}
\renewcommand{\headrulewidth}{0pt}
\fancyfoot[R]{\thepage}

\begin{document} 
\setstretch{1.5}

\begin{titlepage}
    \begin{center}
        \vspace*{\stretch{0.5}}
        \Huge\textbf{Modelo de Pesquisa}
         
        \Large\textit{Edi\c c\~ao Medicina}
         
        \vspace{10cm}
    \end{center}
\end{titlepage}

% Introduction 
\begin{center}
    \Huge\textbf{Introdu\c c\~ao}
\end{center}

% Content
\justifying
Com o passar dos anos, nas nossas vidas chega um momento que decidimos dar
aten\c c\~ao a nossa sa\'ude.  Por cause de complica\c c\~oes que tivemos ou
orque queremos mudar de apar\^encia mas, seja qual for a raz\~ao,  com esta
escolha sempre surgem quest\~oes. Que dieta e protocolos de exerc\'icios
devemos seguir? Que suplementos devemos tomar? Como \'e que estas escolhas
influ\^enciam a qualidade de sono e a energia? E como iremos medir o progresso?

\noindent
Ent\~ao para garantir que as respostas venham de uma fonte de confian\c ca e
para evitar que as decis\~oes sejam impulsivas, pois podem ter um impacto
negativo a longo prazo.  Iremos usar m\'etodos estat\'isticos e de
racionalidade, que fazem parte de um modelo de pesquisa. Este modelo envolve
experi\^encias como a \'unica fonte de verdade e uma colec\c c\~ao de artigos
cient\'ificos como fonte de hip\'oteses que ser\~ao testadas nas
experi\^encias. 



\pagebreak


\begin{center}
    \Huge\textbf{Fonte Veritatis}
\end{center}

\end{document}




